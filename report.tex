\documentclass[a4paper,oneside]{memoir}
\usepackage[english]{babel}
\usepackage[T1]{fontenc}
\usepackage[utf8]{inputenc}
\usepackage{wallpaper}
\usepackage{palatino}
\usepackage{hyperref}
\usepackage{csquotes}
%linespacing
\usepackage{setspace}
\renewcommand{\baselinestretch}{1.5} 

% Bibliography
\usepackage[style=authoryear]{biblatex}
\addbibresource{bib.bib}
\bibliography{bib}

% Promote sections and subsections
\setheadfoot{\onelineskip}{2\onelineskip}
\setheaderspaces{*}{1mm}{*}
% \chapterstyle{plain} % needed?
\checkandfixthelayout

\renewcommand{\thesection}{\arabic{section}}
\makeatletter
\let\l@section\l@chapter
\makeatother

\renewcommand{\thesection}{\arabic{section}}
\renewcommand{\thesubsection}{\thesection.\arabic{subsection}}
\makeatletter
\let\l@subsection\l@section
\let\l@section\l@chapter
\makeatother

% Glossary
\usepackage[numberedsection=nameref]{glossaries}
\renewcommand{\glossarypreamble}{\label{glos}}
\makeglossaries
\newglossaryentry{ai} {
    name = artificial intelligence,
    description = {Artificial intelligence (AI) covers the broad discipline in computer science
that is concerned with replicating intelligent behaviour in computational systems. The exact
definition is controversial for historical reasons \autocite{Nilsson2009}.
}
}
\newglossaryentry{consciousness} {
    name = consciousness,
    description = {Consciousness pertains to the feeling of being alive and attentive.
This circular definition covers over the fact that consciousness is an old and multifaceted
idea that covers many complicated concepts \autocite{sep-consciousness}. In this essay
consciousness will exclusively relate to the \textit{hard} problem of inner experience as
coined by \cite{chalmers1995}, also known as phenomenological consciousness.}
}
\newglossaryentry{top-down} {
    name = top-down,
    description = {This essay employs top-down as a higher-order approach to a solution
or approach to a problem. An example of a top-down approach to understanding
consciousness is the global workspace theory by \autocite{baars2005} or the framework presented
by Francis Crick and Christof Koch \autocite{crick2003}. While both contain elements of 
neurobiology (\gls{bottom-up}) they are explicitly trying to offer an explanation on
what consciousness is}
}
\newglossaryentry{bottom-up} {
    name = bottom-up,
    description = {Bottom-up approaches in this article refer to the combination of
many smaller concepts to form a greater whole. This approach is typical for the natural
sciences. An example of such a bottom-up approach to understanding consciousness is Tononi's
idea of an information integration measure \autocite{Tononi2004}}
}
\newglossaryentry{computation} {
   name = computation,
   description = {Computation refers to any process (in any
substrate) that can deduce new information based on old information. In
this is manifested as computing instructions}
}
\newglossaryentry{ncc} {
   name = {NCC},
   description = {Neural patterns or condition that is minimally sufficient for a conscious
thought to occur. See \autocite{atkinson2000, Hohwy2009}}
}
\newglossaryentry{meme} {
name = meme,
description = {\textit{Meme} is a shortened form of the ancient Greek \textit{mimeme} meaning
'imitated thing' and was coined by Richard Dawkins. A meme refers to a idea or a
\textit{way of behaving} that can be \enquote{copied, transmitted, remembered, taught, shunned,
brandished, ridiculed, parodied, censored, hallowed} \autocite{dennett2017}}
}
\newglossaryentry{tv} {
name = thought vector,
description = {A thought vector is a list of numbers (vector) that describes the attributes
of a state within a neural network \autocite{goh2017} at a specific point in time. A
thought vector thus captures the \textit{thought} of a network at a single instant. One can
imagine how this can be applied to larger neural systems, like mammal brains, to 'capture'
a mental state}
}
\makeglossaries
% Setup captions
%\captionstyle[\centering]{\centering}
%\changecaptionwidth
%\captionwidth{0.8\linewidth}

% Protect against widows and orphans
%\clubpenalty=10000
%\widowpenalty=10000

%\linespread{1.2}

%\raggedbottom

%\chapterstyle{ger}

%\maxsecnumdepth{subsection}

%%  Setup fancy style quotation
%%  ==================================================================
%\usepackage{tikz}
%\usepackage{framed}

%\newcommand*\quotefont{\fontfamily{fxl}} % selects Libertine for quote font

% Make commands for the quotes
%\newcommand*{\openquote}{\tikz[remember picture,overlay,xshift=-15pt,yshift=-10pt]
%     \node (OQ) {\quotefont\fontsize{60}{60}\selectfont``};\kern0pt}
%\newcommand*{\closequote}{\tikz[remember picture,overlay,xshift=15pt,yshift=5pt]
%     \node (CQ) {\quotefont\fontsize{60}{60}\selectfont''};}

% select a colour for the shading
%\definecolor{shadecolor}{rgb}{1,1,1}

% wrap everything in its own environment
%\newenvironment{shadequote}% 
%{\begin{snugshade}\begin{quote}\openquote}
%{\hfill\closequote\end{quote}\end{snugshade}}

%%  Begin document
%%  ==================================================================
\begin{document}
%%  Begin title page
%%  ==================================================================
    \thispagestyle{empty}
    \ULCornerWallPaper{1}{ku-coverpage/samf-farve.pdf}
    \ULCornerWallPaper{1}{ku-coverpage/samf-en.pdf}
    \begin{adjustwidth}{-3cm}{-1.5cm}
    \vspace*{-1cm}
    \textbf{\Huge Feeling of being} \\
    \vspace*{2.5cm} \\
    \textbf{\Huge Consciousness as memes and vectors} \\
    \vspace*{.1cm} \\
    {\huge Analysing consciousness bottom-up and top-down} \\
    \begin{tabbing}
    % adjust the hspace below for the longest author name
    Jens Egholm Pedersen \hspace{1cm} \= \texttt{<xtp778@alumni.ku.dk>} \\
    \\[11cm]

    \textbf{\Large Supervisor} \\
    Claudia Carrara-Augustenborg \hspace{1cm} \texttt{<claudia.augustenborg@psy.ku.dk>} \\
    \end{tabbing}
    \end{adjustwidth}
    \newpage
    \ClearWallPaper
%%  ==================================================================
%%  End title page

\section{Introduction}
\Gls{consciousness} remains an elusive concept despite extensive scrutiny
from many traditions \autocite{sep-consciousness}. Philosophy of mind,
neuroscience, psychology and - recently - computer science have been
prodding the concept from a plenitude of angles.
\footnote{The Stanford Encyclopedia of Philosophy offers an overview of
many of the disciplines and approaches involved in the quest to explain
\gls{consciousness} so far \autocite{sep-consciousness}.}
This endeavour is paramount to understanding the human condition, but
raises unavoidable and intricate existential questions
\autocite{amodei2016, soares2016}.


Since the 19th century developments within neuroscience are rapidly
increasing our understanding of the cognitive processes that partake in
the construction of \gls{consciousness} \autocite{atkinson2000}.\footnote{
Some philosophers require consciousness
to include metaphysical properties \autocite{sep-consciousness} (dualism).
This essay eludes the question by
focusing on falsifiable and positivistic theories, in the hope that
they can bring us closer to the truth - whether that entails dualism
or reductionism.}

The advances and sheer amount of details neuroscience and neuroimaging
techniques led \textcite{atkinson2000} to call for more detailed theories
that identifies \textit{neural correlates of consciousness (\gls{ncc})}.
That call was answered by a plethora of theories
that attempted to infer frameworks, on which novel understandings of
consciousness could be based.\footnote{There are far too many relevant
and interesting papers to list here, but to just mention a few influential examples,
see \cite{baars2005, block2007, crick2003, damasio2003, dehaene2001, graziano2013, kouider2010, Tononi2004, zeki2008, schmidhuber2014, Nilsson2009}.
}

Simultaneous to the advances in neuroscience, \gls{computation}al theory
increased its efforts within \gls{ai} (AI) \autocite{Nilsson2009}, and has
matured to a point where many of the neurophysiological properties can be
replicated in silicon \autocite{Tononi2004, schmidhuber2014, walter2015}.\footnote{Other
technologies show promising advances in forming computational substrates such
as \href{https://en.wikipedia.org/wiki/DNA_computing}{molecular biology}
and \href{https://en.wikipedia.org/wiki/Quantum_computing}{quantum computing},
but have yet to reach the complexity of digital electronic computers.}
The computational prowess of modern digital systems has been shown to grow
with a staggering exponential speed \autocite{Moore1965} - - a development that has proven to
hold since 1965 and that shows no intention of halting
\autocite{Moravec98,kurzweil2001}. If the complexity of the brain can be 
assumed to be finite, this growth will inevitably catch up with the biological
equivalent.\footnote{This is arguably already the case as demonstrated in this
piece of software that simulates an entire worm of the species C.elegans:
\url{http://openworm.org}. Although the worm does not possess advanced cognitive
abilities, this is a proof that biological organisms are not outside the reach
of silicon wafers.}

This last point is important because it tells us that computations can
drastically aid the understanding of consciousness, provided that the
paradigm of neuroscience and computer science find common footing through
\gls{ncc}s. Such a denominator is required because the very definition of
computation depends on a formally defined input before any meaningful output
can be given \autocite{Nilsson2009, schmidhuber2014} (see glossary on
page \pageref{glos}).

Despite the extensive research into \gls{ncc} there is of yet no 'smoking gun'
proving the exact link from neural systems to consciousness
\autocite{sep-consciousness, Hohwy2009, dennett2017}. Many interesting
candidates and perspectives that already accounts for much clinical evidence exist
however.

This essay attempts to follow the advice of \textcite{Hohwy2009}, who advocates
a broader approach to finding \gls{ncc}. Armed with Richard Dawkins' concept of
\textit{\gls{meme}}s \autocite{dennett2017}
and Geoffrey Hintons idea of \textit{\gls{tv}s} \autocite{goh2017}, alternative
conceptualisations of \gls{ncc}s are briefly scrutinized through the lenses of two
contemporary theoretical frameworks for consciousness: global workspace theory by
\textcite{baars2005} and the information integration theory by \textcite{Tononi2004}.

Before proceeding, the reader should be aware that
consciousness in this essay exclusively relates to the \textit{hard} problem of inner experience as
coined by \cite{chalmers1995}, also known as phenomenological consciousness (see glossary on page \pageref{glos}).

\section{Memes and thought vectors}
Assuming an evolutionary viewpoint, any theory of consciousness will have to explain
its added value to evolutional fitness\footnote{Referred by some as \textit{teleofunctionalism} as an opposition to
functionalism, where pain, for instance, would not be seen as a function on the same level
as a limb.} as well as its evolutionary path to achieve that outcome \autocite{dennett2017}. Dennett calls these
the \textit{what for} and \textit{how come} questions, and employs them rigorously on his
journey to explain the concept of consciousness. It requires several layers of indirection to
grasp the evolutionary path to the complex concept of the mind, especially in order to
explain how it could arise in its present form, given the harsh and ruthless principle of natural selection. For this
purpose Dennett invites Dawkins' idea of a \gls{meme}: \enquote{a kind of way of behaving ... that
can be copied, transmitted, remembered, taught, shunned, denounced, brandished ...}
\autocite{dennett2017}. Memes fill the gap between the age of simple organisms and complex
\textit{talking} lifeforms because they are able to answer both the \textit{what for} and
\textit{how come} of how language can evolve gradually.

Instead of requiring a fully functional
uttered word to achieve any form of meaning in an evolutionary context, memes can explain why
thoughts and concepts gained attraction in the race of life: they offer the aid of abstraction
that drastically increases the ability to survive \autocite{dennett2017}. Briefly speaking
a meme is a tool for the brain, just like a hammer is a tool for the body. And like a hammer,
a meme will exist in many variations. Like genetic mutations a large share of those will
be useless, or maybe even harmful, but a fraction will prove to increase the fitness of
the organism carrying the meme, thus ensuring its own survival\footnote{Interestingly
enough this makes memes able to compete for fitness and survival, giving them some form
of autonomy.}

According to the theory of memes an adult brain is then infected with countless memes with
one common denominator: long history of keeping the adult (and the adults lineage) alive.
Consciousness exists amidst all this chaos not as a product, but as a by-product to the
communication between and within the memes themselves \autocite{dennett2017}.

From the perspective of computer science a \gls{meme} is an algorithm: a stepwise solution
to a problem, that can take variation and eventualities into consideration 
\autocite{Nilsson2009, russel2007}. Algorithms in computer science are formalised and described
to great lengths, and have shown to be able to execute in many different computing substrates
\autocite{Nilsson2009}. All algorithms have inputs in one form or another that completely
determine
the behaviour of the code. Such a list of inputs is also known as a vector and can be 
of arbitrary length. 

The recent developments in machine learning is largely relying on vectors to input, transport
and output data \autocite{russel2007}. Such complex learning models are comprised of many smaller
algorithms who, also, depend on vectors. If one were to run a complicated learning model and 
freeze it in time, all the vectors could be captured as a `snapshot' of the model in time.
In other words the model can be represented as a large number of vectors at any given point
in time, only to be resumed, repeated or halted at the whim of the creator \autocite{Nilsson2009}.

This `snapshot' idea has led the deep-learning researcher Geoffrey Hinton at Google to
contemplate a
similar vector for the human mind that would be able to fully capture the state of a consciousness
at a point in time\autocite{goh2017}. Hinton did not further elaborate the idea,
but the technique to take `snapshots' of algorithm is widespread
within the machine learning community \autocite{russel2007}.

Considering memes as algorithms reduces them to a ``black box`` which takes some input
and returns some behaviour. This could, as well as any algorithm, be captured using vectors,
allowing the memes to become the embodiment of the thought vector of the brain. In that light
memes are formalized abstractions for the brain, capable
of executing specified desired behaviour in a highly complex world. Further, this
behaviour does not simply happen out of nowhere, but can be rigorously described in
mathematical terms by capturing all the ``\textit{thought vectors}'' of each individual
\gls{meme} and sub-meme.

\section{The search for NCC}
Guided by years of academic training and tradition, numerous disciplines
have approached the problem of consciousness \gls{top-down} by, in abstract
terms, describing frameworks (like the global workspace theory) 
or \gls{bottom-up} by establishing concepts that can scale to cover abstract
ideas like consciousness \autocite{sep-consciousness}.

The problem with both approaches is that they require a leap of faith in at
least one direction. Bottom-up approaches are forced to mobilise immense methodological
frameworks to establish the connection to something as complicated at phenomenology
and qualia \autocite{sep-consciousness}. Top-down perspectives needs to ratify their
ideas towards the growing corpus of clinical data using some sort of measure that relates
to the higher-order concept, essentially looking for a problem that fits their solution
\autocite{sep-consciousness}.

\section{Testing memes}
The likelihood of solving the puzzle from one direction alone is slim
\autocite{sep-consciousness, Hohwy2009}.

\textcite{Hohwy2009} argues that further progress in the search for \gls{ncc} requires
new approaches to the study of consciousness. He suggests that future work
``targets the presumably causal, mechanistic interplay between content processing and
overall conscious state'' \autocite[p. 436]{Hohwy2009}.

\Gls{meme}s started in practice many million years ago as simple concepts to help the survivability of simpler
organisms, but ended as complicated models and generalisations to assist in navigating and sense-
making in a fast-paced world of constant abstract communication \autocite{dennett2017}.
Drawing upon this, memes can to a large extend be seen as the building blocks for our minds.
Without abstractions to understand collections of items and words humans would never
function in a modern society. 

According to Dennett \gls{meme}s pervades our lives and defines our very language and,
by extension, our mental models with which we see the world \autocite{dennett2017}. If
that is indeed the case it should be examined closer from both perspectives: \gls{top-down} and
\gls{bottom-up}.

\textcite{baars2005} and \textcite{Tononi2004} each posited their own influential theories of
consciousness, representing top-down and bottom-up approaches respectively.
In the following both theories will be assessed in the light of memes and thought
vectors to test its applicability in both domains.

\subsection{Global workspace theory}
\textcite{baars2005} suggests that consciousness exists in a \textit{workspace} of the
brain, whose contents ``activate widespread regions in brain'' \autocite[p.52]{baars2005}.
Baars uses examples from neuroscience on sensory consciousness, working memory, attention
and coma to explain how they all fit with the idea of one large canvas that concerts
many complicated modules to form a unified notion of self.

Baars emphasises that the global workspace enables multiple networks to communicate
with each other in order to solve tasks in concert \autocite{baars2005}. Within his
framework then, is an understanding of vertical complexity, ranging from simpler brain
modules to higher-order conglomerates. This is in line with Dawkins' idea of memes as
a multi-layered and recursive construct, where memes can encompass other memes
\autocite{dennett2017}.

Baars writes that his global workspace theory ``may be thought of as a theater
of mental functioning'', where only one spot on the stage is active at a time
\autocite[p. 46]{baars2005}. The sensory cortices can be activated internally
to perform unconscious processing which, when the signal is strong enough,
triggers a conscious `audience'. Together with S. Franklin, this has even
been implemented as a computer model \autocite{baars2005}, which aligns with 
the computational idea to consciousness of the thought vector approach. 

The global workspace theory emphasises computation and empirical evidence to 
a degree that it is hard to shoot down, while at the same time being too diffuse
in its terms to provide any straightforward methods to falsify its theories.
It remains to be seen, for example, in what way the decentralised information is
gathered in the workspace and what effect that has on consciousness.

Based on the clinical information for the global workspace theory, it is hard
to find significant differences between it and the memes. Both theories operate
on a high level, although Baars seem to elude the specific questions when
it comes to the practicalities of putting his theory to the test.

\subsection{Information integration theory}
Tononi proposes an alternate theory on what consciousness is and how it occurs. He attacks
it by attempting to solve the problem of how to measure consciousness \autocite{Tononi2004}.
Tononi starts out by defining a
measure for how well information is integrated in a network ($\Phi$)
\footnote{By 
\textit{integrated} he refers to how effective information can be relayed and made mutually 
available in a network \autocite{Tononi2004} }
\autocite{Tononi2004}. \\
He then proceeds to equate this information
integration measure with consciousness and proves its saliency by testing it with
several empirical observations.

Tononi writes out the equation for $\Phi$ and thus formalises his theory mathematically.
This aligns with the computational paradigm where even complicated concepts can be
reduced to its discrete constituents. However such a single measure for the 'experience'
of consciousness resonates badly with the multifaceted view of memes. If our
consciousness were to consist of many (competing) memes that constantly fight for
the right to survive, consciousness itself would equal more than just the sum of
its integrated parts.

The approach to memes would also be sceptical to the notion that consciousness scales
with the amount of information. Rather, consciousness would scale with the
efficacy of the memes to construct the side-effect that consciousness (according
to the theory of memes) is. Both perspectives seems to be able to explain the 
neurological evidence mentioned in \textcite{Tononi2004}. To mention a few from the 
memes point of view, split-brain
patients will inevitably experience competing signals since memes per
definition are competing, and the evidence pointing to consciousness in a
distributed thalamocortical network (instead of a single cortical area) can easily
be explained by memes being inherently decentralised (although higher functions
is assumed to happen in the youngest parts of the brain) \autocite{dennett2017, Tononi2004}.

The basic building block of Tononi's theory (information) is consistent with the
thought-vector approach to \gls{meme}s where the information is simply just algorithmic
instructions. However, Tononi never explicitly defines what comprises information in
his article, so the might be unseen discrepancies. On the higher level the theories
start to diverge, perhaps because Tononi does not describe the concepts to great detail.
If the information integration theory is taken to the extreme as presented in the article,
the two theories will likely diverge due to the inherent methodological collectivism
in (recurrent) \gls{meme}s.

\section{Discussion}
Merging the idea of memes with the idea of a thought vector allows for a method 
to approach the idea of higher-level thought, without losing any of the formalism required to
test the theory in practice. This became apparent during the brief examination of
\textcite{Tononi2004} and \textcite{baars2005}, because the theory offered valid
explanations (alternative or aligning) in both cases. Instead of defining a top-down 
or bottom-up concept (like $\Phi$) and scaling it to explain clinical evidences, memes
bridge the gap between the ``non-thinking'' and
``thinking'' substrates by offering a series of stepping stones. 

Rather than dismissing the points of Tononi, memes offers counterproposals that are
significantly easier to test in practice. By sticking to a model where memes
must be autonomous entities with an algorithmic formalisation, memes have an
advantage to many other theories in the literature that have proven near impossible
to verify or falsify \autocite{sep-consciousness}.

The weakness of the theory lies in the vagueness of memes. While it seems more feasible
to formalise the definition of a meme than any other higher-level cognitive theory,
it still remains to be seen how such a definition will look like. It is also not
proven whether the current mathematical cool is capable of expressing the memes to 
a satisfying degree (what is algorithmic steps for a brain?).
Similar to the `minimally
sufficient' condition for consciousness to occur (see glossary on \gls{ncc}) a neural
correlation of memes (NCM) is urgently required.
Although the problem is arguably not as big as finding \gls{ncc}.

\section{Conclusion}
This essay merged the idea of memes with the notion of thought vectors and briefly
applied the concepts to two influential theories of consciousness: global workspace
theory and integrated information theory.

The analysis only scratched the surface of a century old and abstruse subject, but
provided nonetheless a reason for optimism. It was shown that 1) memes can explain parts
of the neurological evidence from the literature, that 2) memes were largely compatible
with the views presented by both \textcite{baars2005} and \textcite{Tononi2004} with 
some added nuances
and finally that 3) memes offer both a \gls{top-down} and \gls{bottom-up} perspective.

The last point is especially interesting because our most powerful computing substrate
(silicon transistors) are excellent \gls{bottom-up} machines, but are ineffective at
abstracting to a level where \gls{top-down} processing is feasible. Memes
allow computational substrate to bootstrap this high-level way of 
\textit{comprehension}, and may be a way to empower computational machines with
the ability to finally adopt new perspectives and augment our capability for
\gls{top-down} thinking - and eventually our consciousness.

A raw and unguided tour de force into AI, void of any form of human consciousness or
ethics, is hardly desirable and the advancements in deep learning further stresses
the search for a proven theory of conscious and \gls{ncc}s.

\clearpage

\printglossary[Glossary]

\printbibliography



\end{document}
