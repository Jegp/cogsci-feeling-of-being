\documentclass[a4paper,oneside]{memoir}
\usepackage[english]{babel}
\usepackage[T1]{fontenc}
\usepackage[utf8]{inputenc}
\usepackage{wallpaper}
\usepackage{palatino}
\usepackage{hyperref}
\usepackage{csquotes}
% Bibliography
\usepackage[style=authoryear]{biblatex}
\addbibresource{bib.bib}
\bibliography{bib}


% Glossary
\usepackage[numberedsection=nameref]{glossaries}
\renewcommand{\glossarypreamble}{\label{glos}}
\makeglossaries
\newglossaryentry{ai} {
    name = artificial intelligence,
    description = {Artificial intelligence (AI) covers the broad discipline in computer science
that is concerned with replicating intelligent behaviour in computational systems. The exact
definition is controversial for historical reasons \autocite{Nilsson2009}.
}
}
\newglossaryentry{consciousness} {
    name = consciousness,
    description = {Consciousness pertains to the feeling of being alive and attentive.
This circular definition covers over the fact that consciousness is an old and multifaceted
idea that covers many complicated concepts \autocite{sep-consciousness}. In this essay
consciousness will exclusively relate to the \textit{hard} problem of inner experience as
coined by \cite{chalmers1995}, also known as phenomenological consciousness.}
}
\newglossaryentry{top-down} {
    name = top-down,
    description = {This essay employs top-down as a higher-order approach to a solution
or approach to a problem. An example of a top-down approach to understanding
consciousness is the global workspace theory by \autocite{baars2005} or the framework presented
by Francis Crick and Christof Koch \autocite{crick2003}. While both contain elements of 
neurobiology (\gls{bottom-up}) they are explicitly trying to offer an explanation on
what consciousness is}
}
\newglossaryentry{bottom-up} {
    name = bottom-up,
    description = {Bottom-up approaches in this article refer to the combination of
many smaller concepts to form a greater whole. This approach is typical for the natural
sciences. An example of such a bottom-up approach to understanding consciousness is Tononi's
idea of an information integration measure \autocite{Tononi2004}}
}
\newglossaryentry{computation} {
   name = computation,
   description = {Computation refers to any process (in any
substrate) that can deduce new information based on old information. In
this is manifested as computing instructions}
}
\newglossaryentry{ncc} {
   name = {NCC},
   description = {Neural patterns or condition that is minimally sufficient for a conscious
thought to occur. See \autocite{atkinson2000, Hohwy2009}}
}
\newglossaryentry{meme} {
name = meme,
description = {\textit{Meme} is a shortened form of the ancient Greek \textit{mimeme} meaning
'imitated thing' and was coined by Richard Dawkins. A meme refers to a idea or a
\textit{way of behaving} that can be \enquote{copied, transmitted, remembered, taught, shunned,
brandished, ridiculed, parodied, censored, hallowed} \autocite{dennett2017}}
}
\newglossaryentry{tv} {
name = thought vector,
description = {A thought vector is a list of numbers (vector) that describes the attributes
of a state within a neural network \autocite{goh2017} at a specific point in time. A
thought vector thus captures the \textit{thought} of a network at a single instant. One can
imagine how this can be applied to larger neural systems, like mammal brains, to 'capture'
a mental state}
}
\makeglossaries
% Setup captions
%\captionstyle[\centering]{\centering}
%\changecaptionwidth
%\captionwidth{0.8\linewidth}

% Protect against widows and orphans
%\clubpenalty=10000
%\widowpenalty=10000

%\linespread{1.2}

%\raggedbottom

%\chapterstyle{ger}

%\maxsecnumdepth{subsection}

%%  Setup fancy style quotation
%%  ==================================================================
%\usepackage{tikz}
%\usepackage{framed}

%\newcommand*\quotefont{\fontfamily{fxl}} % selects Libertine for quote font

% Make commands for the quotes
%\newcommand*{\openquote}{\tikz[remember picture,overlay,xshift=-15pt,yshift=-10pt]
%     \node (OQ) {\quotefont\fontsize{60}{60}\selectfont``};\kern0pt}
%\newcommand*{\closequote}{\tikz[remember picture,overlay,xshift=15pt,yshift=5pt]
%     \node (CQ) {\quotefont\fontsize{60}{60}\selectfont''};}

% select a colour for the shading
%\definecolor{shadecolor}{rgb}{1,1,1}

% wrap everything in its own environment
%\newenvironment{shadequote}% 
%{\begin{snugshade}\begin{quote}\openquote}
%{\hfill\closequote\end{quote}\end{snugshade}}

%%  Begin document
%%  ==================================================================
\begin{document}
%%  Begin title page
%%  ==================================================================
    \thispagestyle{empty}
    \ULCornerWallPaper{1}{ku-coverpage/samf-farve.pdf}
    \ULCornerWallPaper{1}{ku-coverpage/samf-en.pdf}
    \begin{adjustwidth}{-3cm}{-1.5cm}
    \vspace*{-1cm}
    \textbf{\Huge Feeling of being} \\
    \vspace*{2.5cm} \\
    \textbf{\Huge Dimensionality of consciousness} \\
    \vspace*{.1cm} \\
    {\huge On measuring experience} \\
    \begin{tabbing}
    % adjust the hspace below for the longest author name
    Jens Egholm Pedersen \hspace{1cm} \= \texttt{<xtp778@alumni.ku.dk>} \\
    \\[2cm]
    \textbf{\Large Abstract} \\
    Modern developments ...
    \\[6cm]

    \textbf{\Large Supervisor} \\
    Claudia Carrara-Augustenborg \hspace{1cm} \texttt{<claudia.augustenborg@psy.ku.dk>} \\
    \end{tabbing}
    \end{adjustwidth}
    \newpage
    \ClearWallPaper
%%  ==================================================================
%%  End title page

\chapter{Introduction}
\Gls{consciousness} remains an elusive concept despite extensive scrutiny
from many traditions \autocite{sep-consciousness}. Philosophy of mind,
neuroscience, psychology and - recently - computer science have been
prodding the concept from a plenitude of angles.
\footnote{The Stanford Encyclopedia of Philosophy offers an overview of
many of the disciplines and approaches involved in the quest to explain
\gls{consciousness} so far \autocite{sep-consciousness}.}
This endeavour is paramount to understanding the human condition, but
raises unavoidable and highly intricate existential questions
\autocite{amodei2016, soares2016}.


Since the 19th century developments within neuroscience are rapidly
increasing our understanding of the cognitive processes that partake in
the construction of \gls{consciousness} \autocite{atkinson2000}.\footnote{
Some philosophers require consciousness
to include metaphysical properties \autocite{sep-consciousness} (dualism).
This essay eludes the question by
focusing on falsifiable and positivistic theories, in the hope that
they can bring us closer to the truth - whether that entails dualism
or reductionism.}

The advances and sheer amount of details neuroscience and neuroimaging
techniques led \textcite{atkinson2000} to call for more detailed theories
that identifies \textit{neural correlates of consciousness (\gls{ncc})}
\autocite{atkinson2000}. That call was answered by a plethora of theories
that attempted to infer frameworks, on which novel understandings of
consciousness could be based.\footnote{There are far too many relevant
and interesting papers to list here, but to just mention a few influential examples,
see \cite{baars2005, block2007, crick2003, damasio2003, dehaene2001, graziano2013, kouider2010, Tononi2004, zeki2008, schmidhuber2014, Nilsson2009}.
}

Simultaneous to the advances in neuroscience, \gls{computation}al theory
increased its efforts within \gls{ai} (AI) \autocite{Nilsson2009}, and has
matured to a point where many of the neurophysiological properties can be
replicated in silicon \autocite{Tononi2004, schmidhuber2014, walter2015}.\footnote{Other
technologies show promising advances in forming computational substrates such
as \href{https://en.wikipedia.org/wiki/DNA_computing}{molecular biology}
and \href{https://en.wikipedia.org/wiki/Quantum_computing}{quantum computing},
but have yet to reach the complexity of digital electronic computers.}
The computational prowess of modern digital systems has been shown to grow
with a staggering exponential speed \autocite{Moore1965} - - a development that has proven to
hold since 1965 and that shows no intention of halting
\autocite{Moravec98,kurzweil2001}. If the complexity of the brain can be 
assumed to be finite, this growth will inevitably catch up with the biological
equivalent.\footnote{This is arguably already the case as demonstrated in this
piece of software that simulates an entire worm of the species C.elegans:
\url{http://openworm.org}. Although the worm does not possess advanced cognitive
abilities, this is a proof that biological organisms are not outside the reach
of silicon wafers.}

This last point is important because it tells us that computations can
drastically aid the understanding of consciousness, provided that the
paradigm of neuroscience and computer science find common footing through
\gls{ncc}s. Such a denominator is required because the very definition of
computation depends on a formally defined input before any meaningful output
can be given \autocite{Nilsson2009, schmidhuber2014} (see also glossary on
page \pageref{glos}).

Despite the extensive research into \gls{ncc} there is of yet no 'smoking gun'
proving the exact link from neural systems to consciousness
\autocite{sep-consciousness, Hohwy2009, dennett2017}. 

This essay attempts to follow the advice of \textcite{Hohwy2009}, who advocates
a broader approach to finding \gls{ncc}. Armed with Richard Dawkins' concept of
\textit{\gls{meme}}s \autocite{dennett2017}
and Geoffrey Hintons idea of \textit{\gls{tv}s} \autocite{goh2017}, alternative
conceptualisations of \gls{ncc}s are scrutinized using existing theoretical
frameworks of consciousness.


Before proceeding, the reader should be aware of this essay
consciousness will exclusively relate to the \textit{hard} problem of inner experience as
coined by \cite{chalmers1995}, also known as phenomenological consciousness.
\section{Vectors and dimensionality}
what for
how come

\section{Alternative approaches}
Common for each contribution is a fundamental desire to deepen the
understanding of both consciousness as a concept, and the principles
partaking in the creation of the concept. 
Guided by years of academic training and tradition, each discipline have
approached this \gls{top-down} by, in abstract terms, describing 
or \gls{bottom-up} \autocite{sep-consciousness},
: either constructing abstract
frameworks \autocite{block2007,kouider2010} or 

A raw and unguided tour de force into AI void of any form of human consciousness or
ethics is hardly desirable, and, putting the moral existential and moral dilemmas aside,
these advancements further stresses the search for \gls{ncc}s.

\gls{top-down} \textit{and} \gls{bottom-up} \autocite{dehaene2001, baars2005}
\section{Convergent theory}

Some researchers even call for new experimental approaches, because the current 
\clearpage

\printglossary[Glossary]

\printbibliography



\end{document}
