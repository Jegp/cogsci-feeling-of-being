\documentclass[a4paper,oneside]{memoir}
\usepackage[english]{babel}
\usepackage[T1]{fontenc}
\usepackage[utf8]{inputenc}
\usepackage{wallpaper}
\usepackage{palatino}
\usepackage{csquotes}
% Bibliography
\usepackage[style=authoryear]{biblatex}
\addbibresource{bib.bib}
\bibliography{bib}


% Glossary
\usepackage{glossaries}
\newglossaryentry{top-down} {
    name = top-down,
    description = {This essay employs top-down as a higher-order approach to a solution
or approach to a problem. An example of a top-down approach to understanding
consciousness is the global workspace theory by \autocite{baars2005} or the framework presented
by Francis Crick and Christof Koch \autocite{crick2003}. While both contain elements of 
neurobiology (\gls{bottom-up}) they are explicitly trying to explain and evaluate consciousness}
}
\newglossaryentry{bottom-up} {
    name = bottom-up,
    description = {Bottom-up approaches in this article refer to the combination of
many smaller concepts to form a greater whole. This approach is typical for the natural
sciences. An example of such a bottom-up approach to understanding consciousness is Tononi's
idea of an information integration measure \autocite{Tononi2004}}
}
\makeglossaries
% Setup captions
%\captionstyle[\centering]{\centering}
%\changecaptionwidth
%\captionwidth{0.8\linewidth}

% Protect against widows and orphans
%\clubpenalty=10000
%\widowpenalty=10000

%\linespread{1.2}

%\raggedbottom

%\chapterstyle{ger}

%\maxsecnumdepth{subsection}

%%  Setup fancy style quotation
%%  ==================================================================
%\usepackage{tikz}
%\usepackage{framed}

%\newcommand*\quotefont{\fontfamily{fxl}} % selects Libertine for quote font

% Make commands for the quotes
%\newcommand*{\openquote}{\tikz[remember picture,overlay,xshift=-15pt,yshift=-10pt]
%     \node (OQ) {\quotefont\fontsize{60}{60}\selectfont``};\kern0pt}
%\newcommand*{\closequote}{\tikz[remember picture,overlay,xshift=15pt,yshift=5pt]
%     \node (CQ) {\quotefont\fontsize{60}{60}\selectfont''};}

% select a colour for the shading
%\definecolor{shadecolor}{rgb}{1,1,1}

% wrap everything in its own environment
%\newenvironment{shadequote}% 
%{\begin{snugshade}\begin{quote}\openquote}
%{\hfill\closequote\end{quote}\end{snugshade}}

%%  Begin document
%%  ==================================================================
\begin{document}
%%  Begin title page
%%  ==================================================================
    \thispagestyle{empty}
    \ULCornerWallPaper{1}{ku-coverpage/samf-farve.pdf}
    \ULCornerWallPaper{1}{ku-coverpage/samf-en.pdf}
    \begin{adjustwidth}{-3cm}{-1.5cm}
    \vspace*{-1cm}
    \textbf{\Huge Feeling of being} \\
    \vspace*{2.5cm} \\
    \textbf{\Huge Dimensionality of consciousness} \\
    \vspace*{.1cm} \\
    {\huge On measuring experience} \\
    \begin{tabbing}
    % adjust the hspace below for the longest author name
    Jens Egholm Pedersen \hspace{1cm} \= \texttt{<xtp778@alumni.ku.dk>} \\
    \\[2cm]
    \textbf{\Large Abstract} \\
    Modern developments ...
    \\[6cm]

    \textbf{\Large Supervisor} \\
    Claudia Carrara-Augustenborg \hspace{1cm} \texttt{<claudia.augustenborg@psy.ku.dk>} \\
    \end{tabbing}
    \end{adjustwidth}
    \newpage
    \ClearWallPaper
%%  ==================================================================
%%  End title page

\chapter{Introduction}


\autocite{dennett2017}.

what for
how come

\section{Vectors and dimensionality}

\section{Alternative approaches}
Consciousness remains an elusive concept despite extensive scrutiny
from many traditions \autocite{sep-consciousness}. Philosophy of mind,
neuroscience, psychology and - recently - computer science have been
prodding the concept from a plenitude of angles.
\footnote{Just to name a few, see \autocite{block2007, sep-consciousness, schmidhuber2014,
Nilsson2009}.}
Common for each contribution is a fundamental desire to deepen the
understanding of both consciousness as a concept, and the principles
partaking in the creation of the concept. 
Guided by years of academic training and tradition, each discipline have
approached this \gls{top-down} by, in abstract terms, describing 
or \gls{bottom-up} \autocite{sep-consciousness},
: either constructing abstract
frameworks \autocite{block2007,kouider2010} or 

\gls{top-down} \textit{and} \gls{bottom-up} \autocite{dehaene2001, baars2005}
\section{Convergent theory}

\clearpage

\printglossaries

\printbibliography



\end{document}
